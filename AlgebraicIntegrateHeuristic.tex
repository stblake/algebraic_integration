%
\documentclass[12pt]{article}  
\usepackage{color,amssymb,amsthm,amsmath,amsxtra,amsfonts,breqn,
fullpage,graphicx,hyperref,float,appendix,mathtools,eqparbox, 
lscape,pdfpages,comment}

% \usepackage{euler}
% \usepackage{times}
% \usepackage{mathptmx}

\usepackage[latin1]{inputenc}
\usepackage{tikz}
\usetikzlibrary{shapes,arrows}

\interfootnotelinepenalty=100000 % force TeX for not break footnotes across pages.

\newcommand{\overbar}[1]{\mkern 1.5mu\overline{\mkern-1.5mu#1\mkern-1.5mu}\mkern 1.5mu}

\usepackage{fancyhdr}
\pagestyle{fancy} % enable fancy page style
\renewcommand{\headrulewidth}{0pt} % comment if you want the rule
\fancyhf{} % clear header and footer
\fancyhead[RO]{\thepage}   
\fancyhead[LE]{\thepage} 

\usepackage{xcolor}
\definecolor{dark-red}{rgb}{0.4,0.15,0.15}
\definecolor{dark-blue}{rgb}{0,0,0.45}
\hypersetup{colorlinks, linkcolor={dark-blue},citecolor={blue}, urlcolor={blue}}

\numberwithin{equation}{section}
\usepackage{pdflscape}

\definecolor{gold}{rgb}{0.85,0.66,0.0}
\def\todo#1{\textcolor{red}{\textbf{**** TODO -- #1 ****}}}
\def\writeme{\textcolor{gold}{\textbf{WRITE ME!!}}}
\def\rewrite{\textcolor{blue}{\textbf{RE-WRITE ME!!}}}


\theoremstyle{definition}
\newtheorem{example}{Example}[section]
\newtheorem{definition}{Definition}

\usepackage{etoolbox}
\makeatletter
\providecommand{\institute}[1]{% add institute to \maketitle
  \apptocmd{\@author}{\end{tabular}
    \par
    \begin{tabular}[t]{c}
    #1}{}{}
}
\makeatother

\usepackage[ddmmyyyy,hhmmss]{datetime}

\begin{document}

\title{A Simple Method for Computing Some Pseudo-Elliptic Integrals in Terms of Elementary Functions}
\author{Sam Blake}
\institute{\textit{The University of Melbourne}}
\date{DRAFT version: \today:\currenttime}
\maketitle

\begin{abstract}
We introduce a method for computing some pseudo-elliptic integrals in terms of elementary 
functions. The method is simple and fast in comparison to the algebraic case of the 
Risch-Trager-Bronstein algorithm\cite{Risch1969}\cite{Trager1984}\cite{Bronstein1990}. 
This method can quickly solve many pseudo-elliptic integrals, which other well-known 
computer algebra systems (CAS) either fail, return an answer in terms of special functions, 
or require more than 20 seconds of computing time. Unlike the symbolic integration algorithms 
of Risch\cite{Risch1969}, Davenport\cite{Davenport1979}, Trager\cite{Trager1984}, Bronstein\cite{Bronstein1990} 
and Miller\cite{Miller2012}; our method is not a decision process. The implementation 
of this method is less than 200 lines of Mathematica code and can be easily ported 
to other CAS that can solve systems of linear equations.
\end{abstract}

%\todo{Comment on integrals we can and cannot do.}\\

%\todo{Comment on completeness. Should be exhaustive up to the degree bounds on the substitutions and rational forms.}\\

%\todo{Comment on branch cuts and how we try to get around it.}\\

\section{Introduction}
The problem of finding elementary solutions to integrals of algebraic functions has 
challenged mathematicians for centuries. In 1905, Hardy conjectured that the problem may be 
unsolvable\cite{Hardy1916}. Sophisticated algorithms have been developed, including 
the famed Risch algorithm\cite{Risch1969} and its modern variants by Davenport\cite{Davenport1979}, 
Trager\cite{Trager1984}, Bronstein\cite{Bronstein1990} and Miller\cite{Miller2012}. However, 
it is known that the implementation of these algorithms are highly complex and sometimes 
fail, are incomplete or hang\cite{fricas_risch_status}. \\

%\todo{Comment on the completeness of Risch implementations in various CAS. Can we find out why 
%part of Risch was not implemented in AXIOM?} \\

%\todo{An excellent introduction to the Risch algorithm, and in particular the transcendental 
%case of the Risch algorithm can be found in Geddes\cite{Geddes1992}}\\

We will be describing a seemingly new method for computing pseudo-elliptic integrals. 
\begin{definition}
For our purposes, a pseudo-elliptic integral is of the form 
$$\int \frac{a(x)}{b(x)}p(x)^{n/m}dx,$$
where $a(x),b(x),p(x)$ are polynomials, $\deg_x(p(x))>2$, $\gcd(n,m)=1.$
\end{definition} 

Before we describe our method, we begin with some background and motivating examples. \\

The \textit{derivative divides} method is a substitution method that finds all composite 
functions, $u=g(x)$, of the integrand $f(x)$, and tests if $f(x)$ divided by the derivative of 
$u$ is independent of $x$ after the substitution of $u=g(x)$. In other words, up to a constant
factor the derivative divides method simplifies integrals of the form $\int f(g(x))g'(x)dx$ to 
$\int f(u) \, du$. The following integral illustrates the derivative divides method 
$$\int x^2\sqrt{1+x^3}dx=\int \frac{\sqrt{u}}{3} \, du,$$ 
where $u=1+x^3$ and $du=3x^2dx$. This method was first implemented in Moses symbolic integrator, 
SIN, in 1967\cite{Moses1967} and is used in some CAS prior to calling more advanced 
algorithms\cite[pp. 473-474]{Geddes1992}.\\

A slightly more difficult example where the derivative divides method fails is 
$$\int \sqrt{x-1+\sqrt{x-1}} \, dx.$$
In this case we make the substitution $u=\sqrt{x-1}$, then $2udu=dx$. Furthermore, we need to 
express $x-1$ in terms of $u$, which is $u^2=x-1$. Then the integral becomes
$$\int 2u\sqrt{u^2+u}\,du.$$
While this integral was somewhat more difficult than the previous example, we could still 
pick a composite function, $\sqrt{x-1}$, of our integrand, $\sqrt{x-1+\sqrt{x-1}},$ to use 
as our $u$ substitution. In the following well-known example, this is not immediately possible
$$\int \frac{x^2-1}{\left(x^2+1\right)\sqrt{1+x^4}} \, dx.$$
A common approach to solve this integral is rearranging the integrand into
$$\int \frac{x^2\left(1-1\left/x^2\right.\right)}{x(x+1/x)\sqrt{x^2\left(x^2+1\left/x^2\right.\right)}} \, dx = 
\int \frac{1-1\left/x^2\right.}{(x+1/x)\sqrt{(x+1/x)^2-2}} \, dx. $$ 
Then the substitution $u=x+1/x$, $du=\left(1-1\left/x^2\right.\right)dx$ yields the integral
$$\int \frac{du}{u\sqrt{u^2-2}},$$
which can be transformed into a rational function using the Euler substitution\cite{Euler}.\\

Moses devised methods for integrating specific classes of integrals, these 
included $$\int R(x) \exp(p(x)) dx,$$ where $R(x)$ is a rational function of $x$ and $p(x)$ 
is a polynomial in $x$\cite[pp. 85]{Moses1967}. Shortly after, these methods were 
phased-out with the work of Risch. However in most CAS the algebraic case of the Risch 
algorithm is either partially implemented, not implemented, or contains computational 
bottlenecks that result in long computations. So for algebraic functions a domain-specific 
approach still has merit. 

\section{A canonical form for algebraic functions}
We require that the integrand is of the form $\frac{p(x)}{q(x)}r(x)^{n/m}$, where 
$p(x),q(x),r(x)\in \mathbb{Z}[x]$ and $\gcd (n,m)=1$. If the integrand is not in this form 
then we attempt to write it in this form plus a rational function of $x$. For example, 
$$\frac{x^2+1+x\sqrt[4]{x^4+1}}{\left(x^2+1\right) \sqrt[4]{x^4+1}} = 
\frac{y^3}{x^4+1}+\frac{x}{x^2+1},$$
where $y^4 = x^4+1$. The rational part, $\frac{x}{x^2+1},$ is integrated using known algorithms for rational 
function integration\cite{Bronstein1997}. An algorithmic way to create this representation is given in 
Trager\cite{Trager1984}, however it requires an integral basis and is beyond the scope of this paper. \\

\section{A method for solving some pseudo-elliptic integrals}

Following on from our example integral $\int \frac{x^2-1}{\left(x^2+1\right)\sqrt{1+x^4}} \, dx$, 
where making the substitution $u=x+1/x$ resulted in the integral $\int \frac{du}{u\sqrt{u^2-2}}$. 
We would like to generalise this method, however the difficulty was in the choice of the algebraic 
manipulation to the form 
$\int\frac{1-1\left/x^2\right.}{(x+1/x)\sqrt{(x+1/x)^2-2}} \, dx$ in order to discover a rational 
substitution, which simplifies the integral. Consequently our approach does not directly rely 
on such an algebraic manipulation of the integrand. \\

Our method attempts to parameterise constants $a_0, a_1, a_2$, polynomials $a(u), b(u)$ and a 
substitution of the form 
$$u = \frac{s(x)}{x^k}$$ 
such that
\begin{equation}
\int \frac{p(x)}{q(x)}r(x)^{n/m}dx = \int \frac{a(u)}{b(u)}\left(a_2u^2+a_1u+a_0\right)^{n/m}du,\label{reduction}
\end{equation}
where $\deg_x(r(x))>2$ and $\gcd (n,m)=1$. Consequently, our method does not directly compute the 
integral, and requires a recursive call to an algebraic integrator\footnote{If such an integrator is not 
available then a reasonable implementation could call a lookup table of algebraic forms\cite{Prudnikov1986} 
followed by a rational function integrator\cite{Bronstein1997}.}. We note that a reduction to this form
does not guarantee an elementary solution (for example, when $m>2,a_2\ne0$ an elementary form 
is often not possible). \\

Our method is broken into two parts. The first part is computing the \textit{radicand part of 
the integral}, which is a parameterisation of $a_0,a_1,a_2$ and the $u$ 
substitution. The second part is computing the \textit{rational part of the integral}, which is 
a parameterisation of $a(u)$ and $b(u)$.\\

\textit{The radicand part of the integral.} Clearly, if we cannot parameterise the radicand $r(x)$ 
to the form $a_2u^2+a_1u+a_0$ for a given substitution, then we cannot find a 
parameterisation of (\ref{reduction}). Thus, we begin by computing the radicand part 
of the integral, which requires solving 
$$r(x) = \text{num}\left( a_2u^2 + a_1u + a_0 \right),$$
for the constants $a_0,a_1,a_2$ and the substitution $u = s(x)/x^k$. We do this by iterating over 
$0 < d \leq N$ such that $s(x) = \sum\limits_{i=0}^{d} c_i\,x^{i}$, where for each $d$ we 
iterate over $0 < h \leq M$ such that $u = s(x)/x^h$. Given a candidate $u$, we then iterate over 
the radicands 
$$r_1(u) = a_1 u + a_0,\quad r_2(u) = a_2 u^2 + a_0, \quad r_3(u) = a_2 u^2 + a_1 u + a_0,$$ 
where for each radicand we require $m \mid \text{deg}_x\left(\text{den}\left(r_i(u)\right)\right)$ 
(otherwise we would have a fractional power in the denominator of $r_i(u)^{n/m}$), then we solve 
\begin{equation}
r(x) = \text{num}\left(r_i(u)\right).\label{reduction_radicand}
\end{equation}
If $\text{deg}_x\left(r(x)\right) \ne \text{deg}_x\left( \text{num}\left(r_i(u)\right) \right)$, then 
a solution does not exist (as we have already considered lower-degree polynomials).
Otherwise we solve (\ref{reduction_radicand}) by equating coefficients of $x$ and solving the system of 
equations for the unknowns $a_0$, $a_1$, $a_2$, $c_0$, $c_1$, $\cdots$, $c_d$. If a solution 
(or multiple solutions) exists we move to computing the rational part of the integral, otherwise 
we move onto the next radicand or candidate substitution. If no solution exists to the radicand 
part of the integral for any candidate $u$-substitutions, then our method fails to 
compute the integral.\\

\textit{The rational part of the integral.} Given the substitution and solution set of 
the radicand part of the integral, we now look to solve the rational part of the integral, 
which is given by 
\begin{equation}
\frac{p(x)}{q(x)} = \frac{a(u)}{b(u)}\frac{u'(x)}{\text{den}\left(a_2u^2+a_1u+a_0\right)^{n/m}}, \label{reduction_rational}
\end{equation}
where $a_0$, $a_1$, $a_2$, $u(x)$ are known and $a(u)$, $b(u)$ are unknown. The degree bound estimate  
of $a(u)$ and $b(u)$ is given by $\mathcal{D} = \deg_x(u(x)) + \deg_x(u'(x)) + \max\left( \deg_x(p(x)), \deg_x(q(x)) \right)$. We solve 
(\ref{reduction_rational}) by increasing the degree, $d$, of $a(u)$ and $b(u)$ from 1 to the degree 
bound, $\mathcal{D}$, where for each iteration we solve
\begin{equation*}
p(x)\,b(u(x))\,\text{den}\left(a_2u^2 + a_1u + a_0\right)^{n/m} - q(x)\,a(u(x))\,u'(x) = 0,
\end{equation*}
where $a(u) = \sum\limits_{i=0}^d v_i u^i$, $b(u) = \sum\limits_{i=0}^d v_{d+i+1} u^i$, and 
$a(u(x))$, $b(u(x))$ are rational functions in $x$ after replacing $u$ with the candidate 
substitution. As before, we equate powers of $x$ and solve for the unknowns $v_0$, $v_1$, 
$\cdots$, $v_{2d-1}$. If a solution is found, then we have a complete solution to 
(\ref{reduction}) and we stop. Otherwise if we have iterated up to the degree bound, and 
iterated through all solution sets from the radicand part of the integral and we have not 
computed a solution, then the candidate substitution is rejected and must return to the 
radicand part of the integral to try the next substitution.
\iffalse
\section{Examples}
We will now present four detailed examples. 
\fi
\iffalse
\begin{example}
As a simple example, consider the integral $$\int \frac{x^8-1}{x^3 \sqrt{x^8+1}} \, dx.$$ The 
canonical form is $$\int \frac{\left(x^8-1\right)\sqrt{x^8+1}}{x^3\left(x^8+1\right)}\, dx.$$ 

\noindent\textit{The radicand part of the integral}. We find the substitution 
$u=\left(c_1 x^4+c_0\right)/x^2$ yields a solution to the radicand part of 
the integral, and is parameterised as follows
$$x^8+1 = \text{num}\left( a_2u^2 + a_1u + a_0 \right) = 
a_2 c_1^2 x^8 + a_1 c_1 x^6 + \left(2 a_2 c_0 c_1+a_0\right)x^4 + a_1 c_0 x^2 + a_2 c_0^2.$$
Then, equating coefficients of powers of $x$ yields the system of equations
\begin{align*}
 a_2 c_0^2&=1 \\
 a_1 c_0&=0 \\
 2 a_2 c_0 c_1+a_0&=0 \\
 a_1 c_1&=0 \\
 a_2 c_1^2&=1,
\end{align*}
which has the solution $c_0=-1,c_1=-1,a_0=-2,a_1=0,a_2=1$. Thus, the radicand part of the integral is 
$u^2-2,$ where $u=-\left(x^4+1\right)/x^2$. \\

\noindent\textit{The rational part of the integral}. Now we see if a solution exists to the rational part 
of the integral. The degree bound on the solution to the rational part is 4. If the degree is 
1, we have no solution. If the degree is 2, we have the form 
$$\frac{a(u)}{b(u)}=\frac{v_2u^2+v_1u+v_0}{v_5u^2+v_4u+v_3}.$$
For the rational part, we are solving the following equation
$$\frac{x^8-1}{x^3}=\left(\frac{v_2u^2+v_1u+v_0}{v_5u^2+v_4u+v_3}\right)\text{den}\left(u^2-2\right)^{-1/2}\,u'(x),$$
where $\text{den}\left(u^2-2\right)^{-1/2}=1\left/x^2\right.$. 
After replacing $u$ with $-\left(x^4+1\right)/x^2$ and $u'(x)$ with $2\left(1 - x^4\right)/x^3$, we have
$$\frac{x^8-1}{x^3}
=\frac{2\left(1 - x^4\right) \left(u^2 v_2+u v_1+v_0\right)}{x^5 \left(u^2 v_5+u v_4+v_3\right)}
=\frac{2\left(1 - x^4\right) \left(v_2x^8-v_1 x^6+v_0 x^4+2 v_2 x^4-v_1 x^2+v_2\right)}{x^5 \left(v_5 x^8-v_4 x^6+v_3 x^4+2 v_5 x^4-v_4 x^2+v_5\right)},$$
which after clearing denominators is a polynomial equation in $x$, given by
\begin{multline*}
2 v_2 x^{23}+\left(v_5-2 v_1\right) x^{21}+\left(2 v_0+2 v_2-v_4\right) x^{19}+\left(v_3+2 v_5\right) x^{17}+\left(-2 v_0-v_4\right) x^{15}+\\
\left(2v_0+v_4\right) x^{11}+\left(-v_3-2 v_5\right) x^9+\left(-2 v_0-2 v_2+v_4\right) x^7+\left(2 v_1-v_5\right) x^5-2 v_2 x^3=0
\end{multline*}
and we solve for the undetermined coefficients $v_0,v_1,v_2,v_3,v_4,v_5$. Then equating 
coefficients of powers of $x$ yields the system of equations
\begin{align*}{r}
 -2 v_2&=0 \\
 2 v_1-v_5&=0 \\
 -2 v_0-2 v_2+v_4&=0 \\
 -v_3-2 v_5&=0 \\
 2 v_0+v_4&=0 \\
 -2 v_0-v_4&=0 \\
 v_3+2 v_5&=0 \\
 2 v_0+2 v_2-v_4&=0 \\
 v_5-2 v_1&=0 \\
 2 v_2&=0,
\end{align*}
which has the solution $v_0=0, v_2=0, v_3=-4 v_1, v_4=0, v_5=2 v_1$. Thus, the rational part 
of the integral is
$$\frac{v_1u}{2 v_1u^2-4 v_1}=\frac{u}{2 \left(u^2-2\right)}$$
and the integral is given by
$$\int \frac{x^8-1}{x^3 \sqrt{x^8+1}} \, dx=\int \frac{u\sqrt{u^2-2}}{2 \left(u^2-2\right)} \, du=\frac{1}{2} \sqrt{u^2-2}=\frac{\sqrt{1+x^8}}{2x^2},$$
where the remaining integral was computed with the Euler substitution, $t=-u+\sqrt{-2+u^2},$ which 
results in a rational function $\left(2+t^2\right)/\left(4t^2\right)$.
\end{example}
\fi

\iffalse
\begin{example}
Consider the integral $$\int \frac{\left(x^8+1\right) \left(x^8+x^4-1\right) \sqrt{x^{16}+1}}{x^9 \left(x^8-1\right)} \, dx.$$ 
This integral is already in our canonical form. \\

\noindent\textit{The radicand part of the integral}. We find the substitution 
$u=\left(c_1 x^8+c_0\right)/x^4$ yields a solution to the radicand part of 
the integral, and is parameterised as follows
$$x^{16}+1 = \text{num}\left(a_2u^2+a_1u+a_0\right) = 
a_2 c_1^2 x^{16} + a_1 c_1 x^{12} + \left(2 a_2 c_0 c_1+a_0\right)x^8 + a_1 c_0 x^4 + a_2 c_0^2.$$
Then, equating coefficients of powers of $x$ yields the system of equations
\begin{align*}
 a_2 c_0^2  &= 1\\
 a_1 c_0 &= 0\\
 2 a_2 c_0 c_1+a_0 &= 0\\
 a_1 c_1 &= 0\\
 a_2 c_1^2 &= 1,
\end{align*}
which has the solution $c_0=-1,c_1=1,a_0=2,a_1=0,a_2=1$. Thus, the radicand part of the 
integral is $u^2+2$, where $u=\left(x^8-1\right)/x^4$. \\

\noindent\textit{The rational part of the integral}. Now we see if a solution exists to the rational part of the 
integral. The degree bound on the solution to the rational part is 3. When the degree is 1, we have
$$\frac{a(u)}{b(u)}=\frac{v_1u+v_0}{v_3u+v_2}.$$
For the rational part, we are solving the following equation
$$\frac{\left(x^8+1\right) \left(x^8+x^4-1\right)}{x^9 \left(x^8-1\right)} =
\left(\frac{v_1u+v_0}{v_3u+v_2}\right)\text{den}\left(u^2+2\right)^{-1/2}\,u'(x),$$
where $\text{den}\left(u^2+2\right)^{-1/2}=1\left/x^4\right.$. After replacing $u$ with 
$\left(x^8-1\right)/x^4$ and $u'(x) = \left(4 \left(1+x^8\right)\right)/x^5$, we have
$$\frac{\left(x^8+1\right) \left(x^8+x^4-1\right)}{x^9 \left(x^8-1\right)} = 
\frac{4 \left(x^8+1\right) \left(u v_1+v_0\right)}{x^9 \left(u v_3+v_2\right)} = 
\frac{4\left(x^8+1\right) \left(v_1 x^8+v_0 x^4-v_1\right)}{x^9 \left(v_3 x^8+v_2 x^4-v_3\right)},$$
which after clearing denominators is a polynomial equation in $x$, given by
\begin{multline*}
\left(v_3-4 v_1\right) x^{33}+\left(-4 v_0+v_2+v_3\right) x^{29}+\left(4 v_1+v_2-v_3\right) x^{25}+\\
\left(4 v_1+v_2-v_3\right) x^{17}+\left(4 v_0-v_2-v_3\right) x^{13}+\left(v_3-4 v_1\right) x^9 = 0,
\end{multline*}
which we solve for the undetermined coefficients $v_0,v_1,v_2,v_3$. Then equating coefficients of 
powers of $x$ yields the system of equations
\begin{align*}
 v_3-4 v_1&=0 \\
 4 v_0-v_2-v_3&=0 \\
 4 v_1+v_2-v_3&=0 \\
 4 v_1+v_2-v_3&=0 \\
 -4 v_0+v_2+v_3&=0 \\
 v_3-4 v_1&=0,
\end{align*}
which has the solution $v_0=v_1,v_2=0,v_3=4v_1$. Thus, the rational part of the integral is
$$\frac{u+1}{4u}$$
and the integral is given by
\begin{align*}
&\int \frac{\left(x^8+1\right) \left(x^8+x^4-1\right) \sqrt{x^{16}+1}}{x^9 \left(x^8-1\right)} \, dx\\
&=\int \frac{(1+u) \sqrt{u^2+2}}{4 u} \, du\\
&=\left(\frac{u}{2}+1\right)\sqrt{u^2+2}+\log \left(\sqrt{\frac{u^2}{2}+1}+\frac{u}{\sqrt{2}}\right)-
\sqrt{2} \log \left(\sqrt{2} \sqrt{u^2+2}+2\right)+\sqrt{2}\log (u)\\
&=\frac{\left(\frac{x^8}{8}+\frac{x^4}{4}-\frac{1}{8}\right) \sqrt{x^{16}+1}}{x^8} + 
	\frac{1}{2\sqrt{2}}\log\left(x^8-1\right) - \frac{1}{4} \log \left(2 x^4\right) + \\
&\qquad \frac{1}{4} \log \left(\sqrt{2}\sqrt{x^{16}+1}+\sqrt{2}x^8-\sqrt{2}\right)-
	\frac{1}{2 \sqrt{2}}\log \left(\sqrt{2}\sqrt{x^{16}+1}+2 x^4\right),\\
\end{align*}
where the remaining integral is computed using the Euler substitution, then integrating a  
rational function. 
\end{example}
\fi

\begin{example}
We will apply the method detailed above to compute the following integral
$$\int \frac{\left(x^3-2\right) \sqrt{x^3-x^2+1}}{\left(x^3+1\right)^2} \, dx.$$ 
This integral is already in our canonical form. \\

\noindent\textit{The radicand part of the integral}. We find the substitution 
$u=\left(c_1 x^3+c_0\right)/x^2$ yields a solution to the radicand part of the 
integral, and is parameterised as follows
$$x^3-x^2+1 = \text{num}\left( a_2u^2+a_1u+a_0 \right) = 
	a_2 c_1^2 x^6 + a_1 c_1 x^5 + a_0 x^4 + 2 a_2 c_0 c_1 x^3 + a_1 c_0 x^2 + a_2 c_0^2.$$
Then, equating coefficients of powers of $x$ yields the system of equations
\begin{align*}
 a_2 c_0^2 &= 1\\
 a_1 c_0 &= -1\\
 2 a_2 c_0 c_1 &= 1\\
 a_0 &= 0\\
 a_1 c_1 &= 0\\
 a_2 c_1^2 &= 0,
\end{align*}
which has no solution, so we use a linear radicand in $u$ as follows 
$$x^3-x^2+1 = \text{num}\left( a_1u + a_0 \right) = a_1 c_1 x^3 + a_0 x^2 + a_1 c_0.$$
Again, we equate coefficients of powers of $x$ and we have the system of equations
\begin{align*}
 a_1 c_0 &= 1\\
 a_0 &= -1\\
 a_1 c_1 &= 1,
\end{align*}
which has the solution $a_0 = -1, a_1 = 1, c_0 = 1, c_1 = 1$. Thus, the radicand part of the 
integral is $u-1$, where $u=\left(1+x^3\right)/x^2$. \\

\noindent\textit{The rational part of the integral}. Now we see if a solution exists to the 
rational part of the integral. The degree bound on the solution to the rational part 
is 3. When the degree is 1, we have no solution. When the degree is 2, we have
$$\frac{a(u)}{b(u)}=\frac{v_2u^2+v_1u+v_0}{v_5u^2+v_4u+v_3}.$$
For the rational part, we are solving the following equation
$$\frac{\left(x^3-2\right)}{\left(x^3+1\right)^2} = 
	\left(\frac{v_2u^2+v_1u+v_0}{v_5u^2+v_4u+v_3}\right)\text{den}(u-1)^{-1/2}\,u'(x),$$
where $\text{den}(u-1)^{-1/2}=1/x$. After replacing $u$ with $\left(1+x^3\right)/x^2$ and 
$u'(x)$ with $\left(x^3-2\right)/x^3$ we have
$$\frac{x^3-2}{\left(x^3+1\right)^2}=
\frac{\left(x^3-2\right) \left(u^2 v_2+u v_1+v_0\right)}{x^4 \left(u^2 v_5+u v_4+v_3\right)}=
\frac{\left(x^3-2\right)\left(x^4 v_0+x^2 v_1+x^5 v_1+v_2+2 x^3 v_2+x^6 v_2\right)}{x^4 \left(x^4 v_3+x^2 v_4+x^5 v_4+v_5+2 x^3 v_5+x^6 v_5\right)},$$
which after clearing denominators is a polynomial equation in $x$, given by
\begin{multline*}
-v_2 x^{15}-v_1 x^{14}+\left(v_5-v_0\right) x^{13}+\left(v_4-2 v_2\right) x^{12}
+\left(v_3-v_1\right) x^{11}+\left(2 v_2-v_4\right) x^9+ \\
\left(3v_1-2 v_3\right) x^8+\left(3 v_0-3 v_5\right) x^7+
\left(8 v_2-2 v_4\right) x^6+5 v_1 x^5+\left(2 v_0-2 v_5\right) x^4+7 v_2 x^3+2 v_1 x^2+2 v_2=0,
\end{multline*}
which we solve for the undetermined coefficients $v_0,v_1,v_2,v_3,v_4,v_5$. Then equating coefficients of powers of $x$ yields the system of equations

\begin{align*}
 2 v_2&=0 \\
 2 v_1&=0 \\
 7 v_2&=0 \\
 2 v_0-2 v_5&=0 \\
 5 v_1&=0 \\
 8 v_2-2 v_4&=0 \\
 3 v_0-3 v_5&=0 \\
 3 v_1-2 v_3&=0 \\
 2 v_2-v_4&=0 \\
 -v_1+v_3&=0 \\
 -2 v_2+v_4&=0 \\
 -v_0+v_5&=0 \\
 -v_1&=0 \\
 -v_2&=0,
\end{align*}

which has the solution $v_0=v_5,v_1=0,v_2=0,v_3=0,v_4=0$. Thus, the rational part of the integral is
$$\frac{v_0}{v_0u^2}=\frac{1}{u^2}$$
and the integral is given by 
\begin{multline*}
\int \frac{\left(x^3-2\right) \sqrt{x^3-x^2+1}}{\left(x^3+1\right)^2} \, dx 
=\int \frac{\sqrt{u-1}}{u^2} \, du \\
=-\frac{\sqrt{u-1}}{u}+\tan ^{-1}\left(\sqrt{u-1}\right)
=-\frac{x\sqrt{x^3-x^2+1}}{x^3+1}+\tan ^{-1}\left(\frac{\sqrt{x^3-x^2+1}}{x}\right).\\
\end{multline*}
\end{example}

Our implementation in Mathematica took 0.085 seconds to compute this integral.  

\iffalse
\begin{example}
In our final example we consider the particularly elegant integral 
$$\int\frac{\left(x^2-1\right)\sqrt{x^4+x^2+1}}{\left(x^2+1\right)\left(x^4+x^3+x^2+x+1\right)} \, dx.$$

\todo{Finish this example.}

\end{example}
\fi

\section{Generalisations of the method}

In this section we will look at two generalisations of this method to extend the 
classes of integrals with elementary solutions. We also consider solutions in terms 
of special functions. 

\subsection{Generalisations of the reduction}

This method can be generalised to include reductions of the form 

\begin{align*}
\int \frac{p(x)}{q(x)}r(x)^{n/m}dx &= \int \frac{a(u)}{b(u)}\left(a_2u^{2k}+a_1u^k+a_0\right)^{n/m}du \\
\intertext{and}
\int \frac{p(x)}{q(x)}r(x)^{n/m}dx &= \int \frac{a(u)}{b(u)}\left(a_1u^k+a_0\right)^{n/m}du.\\
\end{align*}

For both these forms we use essentially the same procedure as in the linear or quadratic 
reduction. However, in terms of efficiency, obviously the more forms we try to solve for 
the radicand part of the integral, the longer the process takes to iterate through these 
possible forms. 

\subsection{Non-elementary integrals}

If the integration of the reduced form is permitted to contain special functions and the 
recursive call to the integrator has the capability to integrate algebraic functions 
in terms of special functions, then we can obtain integrals to a much larger class of 
algebraic functions. 

\begin{example}
For the following integral, our method makes the substitution 
$u = \left(2 x^8 - 1\right)/x^2$, to obtain the reduction
\begin{equation*}
\int\frac{6 x^8+1}{\left(2 x^8-1\right) \sqrt[4]{4 x^{16}+2 x^{10}-4 x^8-x^4-x^2+1}}dx = 
\int\frac{du}{2 u \sqrt[4]{u^2+u-1}}.
\end{equation*}
This integral does not have a solution in terms of elementary functions, however Mathematica
can express this integral in terms of the Appell hypergeometric function, 
\href{https://reference.wolfram.com/language/ref/AppellF1.html}{$F_1$}
\begin{multline*}
\int\frac{du}{2 u \sqrt[4]{u^2+u-1}} = \\
-\frac{\sqrt[4]{\frac{1-\sqrt{5}}{2 u}+1} \sqrt[4]{\frac{1+\sqrt{5}}{2 u}+1}}{\sqrt[4]{u^2+u-1}} \times
F_1\left(\frac{1}{2};\frac{1}{4},\frac{1}{4};\frac{3}{2};-\frac{1+\sqrt{5}}{2u},-\frac{1-\sqrt{5}}{2 u}\right).
\end{multline*}
Then the integral is given by 
\begin{multline*}
\int\frac{6 x^8+1}{\left(2 x^8-1\right) \sqrt[4]{4 x^{16}+2 x^{10}-4 x^8-x^4-x^2+1}}dx = \\
-\frac{x \sqrt[4]{\frac{4 x^8-\sqrt{5} x^2+x^2-2}{2 x^8-1}} \sqrt[4]{\frac{4 x^8+\sqrt{5}x^2+x^2-2}{2 x^8-1}}}
{\sqrt{2} \sqrt[4]{4 x^{16}+2 x^{10}-4 x^8-x^4-x^2+1}} \times \\
F_1\left(\frac{1}{2};\frac{1}{4},\frac{1}{4};\frac{3}{2};\frac{\left(-\frac{1}{2}-\frac{\sqrt{5}}{2}\right) x^2}{2 x^8-1},\frac{\left(-\frac{1}{2}+\frac{\sqrt{5}}{2}\right) x^2}{2x^8-1}\right).
\end{multline*}
\end{example}

\subsection{Recursive integration}

In the following example we recursively call our method to obtain an elementary form 
to a difficult integral. 

\begin{example}
For the following integral, our method makes the substitution 
$u = \sqrt{2}\left(2x^8-1\right)/x^2$, then for the recursive integration, we make 
the substitution $u = t^2$ to obtain 
\begin{multline*}
\int \frac{6 x^8+1}{\left(2 x^8-1\right) \sqrt[4]{8 x^{16}-8 x^8-x^4+2}} dx =
 \int\frac{du}{2 u \sqrt[4]{u^2-1}} = \int \frac{dt}{4 t \sqrt[4]{t-1}} = \\
% -\frac{1}{2 \sqrt{2}} \tan ^{-1}\left(1-\sqrt{2} \sqrt[4]{t-1}\right) + 
% \frac{1}{2 \sqrt{2}} \tan ^{-1}\left(\sqrt{2}\sqrt[4]{t-1}+1\right) - 
% \frac{1}{2 \sqrt{2}} \tanh ^{-1}\left(\frac{\sqrt{2}\sqrt[4]{t-1}}{\sqrt{t-1}+1}\right) = \\
% -\frac{1}{2 \sqrt{2}} \tan ^{-1}\left(1-\sqrt{2} \sqrt[4]{u^2-1}\right) + 
% \frac{1}{2 \sqrt{2}} \tan^{-1}\left(\sqrt{2} \sqrt[4]{u^2-1}+1\right) - 
% \frac{1}{2 \sqrt{2}} \tanh ^{-1}\left(\frac{\sqrt{2}\sqrt[4]{u^2-1}}{\sqrt{u^2-1}+1}\right) = \\
\frac{1}{2 \sqrt{2}} \tan ^{-1}\left(\frac{-x^2+\sqrt{8 x^{16}-8 x^8-x^4+2}}{\sqrt{2} x \sqrt[4]{8 x^{16}-8x^8-x^4+2}}\right) - 
\frac{1}{2 \sqrt{2}} \tanh ^{-1}\left(\frac{x^2+\sqrt{8 x^{16}-8x^8-x^4+2}}{\sqrt{2} x \sqrt[4]{8 x^{16}-8 x^8-x^4+2}}\right).
\end{multline*}
\end{example}

In the above example, the integral $$\int\frac{du}{2 u \sqrt[4]{u^2-1}},$$ does not return an elementary 
form in either Mathematica (12.1) or Maple (2018.1).

\section{A comparison with major CAS and algebraic integration packages}

We will compare our method with the Mathematica (12.1.0), Maple (2018.1), REDUCE (5286, 1-Mar-20), 
and FriCAS (version 1.2.6) computer algebra systems. We will also include in the comparison a table 
lookup package, RUBI\cite{Rich2018}, which has been ported to a number of computer algebra systems 
and compares favourably with most built-in integrators on a large suite of problems\cite{rubi_results}. 
We have also included an experimental algebraic integration package developed in Mathematica by Manuel 
Kauers\cite{Kauers2008}. Within the Kauers package we have replaced the calls to Singular in favour
of Mathematica's built-in Groebner basis routine.\\

We have included results from Maple twice. Once with a call of \texttt{int(integrand, x)} 
and once with \texttt{int(convert(integrand, RootOf),x)}. This is because the default 
behaviour of Maple is to not use the Risch-Trager-Bronstein integration 
algorithm\cite{Trager1984}\cite{Bronstein1990} for algebraic functions unless the 
radicals in the integrand are converted to the Maple \texttt{RootOf} notation.\\

Within REDUCE we have used the algint package by James Davenport\cite{Davenport1979}. \\

Our test suite is 190 integrals that can be found on github\cite{test_suite_github}. 
All the integrals in the suite have a solution in terms of elementary functions. \\

We will show the results from all the systems and packages for one integral from the test 
suite. It is intriguing to see the variety of forms for this integral. \\

\noindent Our method returns:
\small
%\begin{multline*}
%\int \frac{\left(x^4-1\right) \sqrt{x^4+1}}{x^8+1} \, dx=
%\frac{1}{4 \sqrt[4]{2}}\log \left(2 x-2^{3/4} \sqrt{x^4+1}\right) - \\
%\frac{1}{4 \sqrt[4]{2}}\log \left(2 x + 2^{3/4} \sqrt{x^4+1}\right) + 
%\frac{1}{2 \sqrt[4]{2}} \tan ^{-1}\left(\frac{\sqrt{x^4+1}}{\sqrt[4]{2} x}\right)
%\end{multline*}
\begin{equation*}
\int \frac{\left(x^4-1\right) \sqrt{x^4+1}}{x^8+1} \, dx= 
-\frac{1}{2 \sqrt[4]{2}}\tan ^{-1}\left(\frac{\sqrt[4]{2} x}{\sqrt{x^4+1}}\right) - 
\frac{1}{2 \sqrt[4]{2}}\tanh^{-1}\left(\frac{\sqrt[4]{2} x}{\sqrt{x^4+1}}\right)
\end{equation*}
\normalsize
FriCAS returns:
\small
\begin{align*}
& \int \frac{\left(x^4-1\right) \sqrt{x^4+1}}{x^8+1} \, dx = \\
& \frac{1}{8 \sqrt[4]{2}}\log\left(\frac{1}{x^8+1}\left(4 x^6+4 x^2+\sqrt{2} \left(x^8+4x^4+1\right)
-\sqrt{x^4+1} \left(2^{3/4} \left(2 x^5+2 x\right)+4 \sqrt[4]{2} x^3\right)\right)\right) - \\
&\frac{1}{8 \sqrt[4]{2}}\log \left(\frac{-1}{x^8+1}\left(4x^6+4 x^2+\sqrt{2} \left(x^8+4 x^4+1\right)+
\sqrt{x^4+1} \left(2^{3/4} \left(2 x^5+2 x\right)+4 \sqrt[4]{2} x^3\right)\right)\right) + \\
& \frac{1}{2\sqrt[4]{2}}\tan ^{-1}\left(\frac{-4 x^6-4 x^2+\sqrt{2} \left(x^8+4 x^4+1\right)}{\sqrt{2} \left(-x^8-1\right)+\sqrt{x^4+1} \left(2^{3/4} \left(2x^5+2 x\right)-4 \sqrt[4]{2} x^3\right)}\right)
\end{align*}
\normalsize
Kauer's algorithm returns: 
\small
$$\int \frac{\left(x^4-1\right) \sqrt{x^4+1}}{x^8+1} \, dx = 
\sum_{512 \alpha ^4-1=0}\alpha  \log \left(4 \alpha  \sqrt{x^4+1}-x\right)$$
\normalsize
Maple (default) returns:
\small
$$\int \frac{\left(x^4-1\right) \sqrt{x^4+1}}{x^8+1} \, dx = 
\frac{1}{2 \sqrt[4]{2}} \tan ^{-1}\left(\frac{\sqrt{x^4+1}}{\sqrt[4]{2} x}\right) - 
\frac{1}{4 \sqrt[4]{2}} \log\left(\frac{\frac{\sqrt{x^4+1}}{\sqrt{2} x}+\frac{1}{\sqrt[4]{2}}}{\frac{\sqrt{x^4+1}}{\sqrt{2} x}-\frac{1}{\sqrt[4]{2}}}\right)$$
\normalsize
Maple (RootOf) returns:
\small
\begin{multline*}
\int \frac{\left(x^4-1\right) \sqrt{x^4+1}}{x^8+1} \, dx = 
\frac{1}{4 \sqrt[4]{2}}\log \left(\frac{2\times2^{3/4} x^4-8 \sqrt{x^4+1} x+4 \sqrt[4]{2} x^2+2\times2^{3/4}}{-2 x^4+2 \sqrt{2} x^2-2}\right) + \\
\frac{i}{4 \sqrt[4]{2}}\log \left(\frac{2\times2^{3/4} i x^4-8 \sqrt{x^4+1} x-4 i \sqrt[4]{2} x^2+2\times2^{3/4} i}{2 x^4+2 \sqrt{2} x^2+2}\right)
\end{multline*}
\normalsize
Mathematica returns:
\small
\begin{multline*}
\int \frac{\left(x^4-1\right) \sqrt{x^4+1}}{x^8+1} \, dx = \frac{1}{2} \sqrt[4]{-1} \left(-2 F\left(\left.i \sinh ^{-1}\left(\sqrt[4]{-1} x\right)\right|-1\right) \right. + \\
\Pi\left(-\sqrt[4]{-1};\left.i \sinh ^{-1}\left(\sqrt[4]{-1} x\right)\right|-1\right) + 
\Pi \left(\sqrt[4]{-1};\left.i \sinh ^{-1}\left(\sqrt[4]{-1} x\right)\right|-1\right) + \\
\left. \Pi\left(-(-1)^{3/4};\left.i \sinh ^{-1}\left(\sqrt[4]{-1} x\right)\right|-1\right)+\Pi \left((-1)^{3/4};\left.i \sinh ^{-1}\left(\sqrt[4]{-1} x\right)\right|-1\right)\right)
\end{multline*}
\normalsize
where \textit{F} is the elliptic integral of the first kind, and $\Pi$ is the incomplete elliptic integral. \\

\noindent REDUCE (using algint package) returns:
\small
$$\int \frac{\left(x^4-1\right) \sqrt{x^4+1}}{x^8+1} \, dx = 
\int \frac{x^4 \sqrt{x^4+1}}{x^8+1} \, dx-\int \frac{\sqrt{x^4+1}}{x^8+1} \, dx$$
\normalsize
RUBI returns\footnote{After posting a preprint of this comparison on the 
\href{https://groups.google.com/forum/\#!forum/sci.math.symbolic}{sci.math.symbolic}
newsgroup, Albert Rich (the creator of RUBI) devised a general rule for integrals of this 
form which will be included in the next release of RUBI: %
$$\int\frac{\left(f+g x^4\right)\sqrt{d+e x^4}}{a+b x^4+c x^8} = 
\frac{e^2 f}{2 c d \sqrt[4]{2 d e-\frac{b e^2}{c}}} \tan ^{-1}\left(\frac{x \sqrt[4]{2 d e-\frac{b e^2}{c}}}{\sqrt{d+e x^4}}\right) 
	+\frac{e^2 f}{2 c d \sqrt[4]{2 d e-\frac{b e^2}{c}}} \tanh ^{-1}\left(\frac{x \sqrt[4]{2 d e-\frac{be^2}{c}}}{\sqrt{d+e x^4}}\right),$$ when $e f+d g=0$ and $c d^2-a e^2=0$.}:
\small
\begin{multline*}
\int \frac{\left(x^4-1\right) \sqrt{x^4+1}}{x^8+1} \, dx=
-\frac{\tan ^{-1}\left(\frac{\sqrt[4]{2} x}{\sqrt{x^4+1}}\right)}{2 \sqrt[4]{2}}-
\frac{\tanh^{-1}\left(\frac{\sqrt[4]{2} x}{\sqrt{x^4+1}}\right)}{2 \sqrt[4]{2}} + \\
\frac{\left(x^2+1\right) \sqrt{\frac{x^4+1}{\left(x^2+1\right)^2}} F\left(2\tan ^{-1}(x)|\frac{1}{2}\right)}{2 \sqrt{x^4+1}} + 
\frac{\left((-1-i)-i \sqrt{2}\right) \left(x^2+1\right) \sqrt{\frac{x^4+1}{\left(x^2+1\right)^2}}
F\left(2 \tan ^{-1}(x)|\frac{1}{2}\right)}{8 \sqrt{x^4+1}} + \\
\frac{\left(\sqrt{2}+(-1+i)\right) i \left(x^2+1\right) \sqrt{\frac{x^4+1}{\left(x^2+1\right)^2}}
F\left(2 \tan ^{-1}(x)|\frac{1}{2}\right)}{8 \sqrt{x^4+1}} + \\
\frac{\left(\sqrt{2}+(1+i)\right) i \left(x^2+1\right) \sqrt{\frac{x^4+1}{\left(x^2+1\right)^2}}
F\left(2 \tan ^{-1}(x)|\frac{1}{2}\right)}{8 \sqrt{x^4+1}} - \\
\frac{\left(\frac{1}{8}-\frac{i}{8}\right) \left(1+(-1)^{3/4}\right) \left(x^2+1\right)
\sqrt{\frac{x^4+1}{\left(x^2+1\right)^2}} F\left(2 \tan ^{-1}(x)|\frac{1}{2}\right)}{\sqrt{x^4+1}}
\end{multline*}
\normalsize
where \textit{F} is the elliptic integral of the first kind, and $\Pi$ is the incomplete elliptic integral. \\

The table below summarises the comparison between all systems on the test suite of integrals. The 
expression size is the leaf count(\texttt{LeafCount}) in Mathematica.

\begin{table}[H]
\centering
\caption{A comparison of our method with major CAS and algebraic integration packages.}
{\small
\begin{tabular}{c|c|c|c|c|c|c}
	\begin{tabular}{@{}c@{}} \text{CAS} \\ \text{/package} \end{tabular} & 
	\begin{tabular}{@{}c@{}} \text{Elementary} \\ \text{forms [\%]} \end{tabular} &
	\begin{tabular}{@{}c@{}} \text{Contains} \\ \text{$\int dx$ [\%]} \end{tabular} &
	\begin{tabular}{@{}c@{}} \text{Contains} \\ \text{special} \\ \text{functions [\%]} \end{tabular} &
	\begin{tabular}{@{}c@{}} \text{Timed-out} \\ \text{($>$20s) [\%]} \end{tabular} &
	\begin{tabular}{@{}c@{}} \text{Median} \\ \text{time [s]} \end{tabular} &
	\begin{tabular}{@{}c@{}} \text{Expression size} \\ \text{- string length} \end{tabular} \\
\hline
 \text{new} & 100.0 & 0.0 & 0.0 & 0.0 & 0.26 & 81--118 \\
\hline
 \text{Maple} (\texttt{RootOf}) & 91.6 & 2.0 & 0.0 & 6.3 & 4.32 & 136--238 \\
\hline
 \text{FriCAS} & 70.0 & 4.7 & 0.0 & 25.3 & 0.26 & .\text{  }--129 \\
\hline
 \text{Kauers} & 62.6 & 7.9 & 0.0 & 29.4 & 0.40 & 88--116 \\
\hline
 \text{RUBI} & 13.7 & 60.0 & 16.3 & 10.0 & 0.34 & 244--405 \\
\hline
 \text{Maple} & 11.6 & 55.3 & 33.2 & 0.0 & 0.34 & 436--943 \\
\hline
 \text{Mathematica} & 9.5 & 44.7 & 43.2 & 2.63 & 1.21 & 574--1019 \\
\hline
 \text{REDUCE} (\text{algint}) & 6.3 & 65.3 & 0.0 & 28.4 & 1.16 & . \\
\end{tabular}
}
\end{table}

In regards to this table: The median time excludes integrals that timed-out. For 
FriCAS and REDUCE we did not find a built-in routine to compute the leaf count. \\

Mathematica is far more comfortable in returning an answer in terms of elliptic 
functions, but these results are far from concise. Computing the integral in 
terms of elliptic functions takes considerable time. \\

Maple fares much better when it uses the algebraic case of the Risch-Trager-Bronstein 
algorithm. When the integrand is converted to \texttt{RootOf} notation in Maple, 
the time required increases significantly, however the results improve significantly. 
We are then left wondering why Maple does not make this conversion internally if 
the initial algebraic integration routines fail? \\

FriCAS is a fork of AXIOM. FriCAS inherits the most complete implementation of the 
Risch-Trager-Bronstein algorithm for integration in finite (elementary) terms. The 
results suggest that the FriCAS (version 1.2.6) integrator is either fast, hangs, or 
returns an error. We have reported these issues to the FriCAS developers and most 
issues will be fixed in the next release\cite{fricas_sci.math.symbolic}. Thus, the 
performance of FriCAS on our test suite of integrals will be much better, with the 
exception of incomplete parts of the Risch-Bronstein-Trager algorithm for algebraic 
functions, for example after 16 seconds of computing time FriCAS (version 1.2.6) returns \\

\footnotesize
\begin{verbatim}
(1) -> integrate(((4+x^6)*(-4+x^4+2*x^6)*
                       (32-14*x^4-32*x^6-4*x^8+7*x^10+8*x^12)^(1/2))/(x^9*(-2+x^6)),x)
 
   >> Error detected within library code:
   integrate: implementation incomplete (residue poly has multiple non-linear factors)
\end{verbatim}
\normalsize

When Kauer's heuristic does not time out, then its fast, with only 11 problems taking 
longer than 2 seconds. It also often returns a concise form. Miller\cite{Miller2012} 
has extended Kauer's heuristic to a complete algorithm for the logarithmic part of 
the integral of a mixed algebraic-transcendental function. Unfortunately we did not 
have access to an implementation for comparison. \\

\section{Conclusions}

We have shown that we can efficiently solve some pseudo-elliptic integrals. Our method compares 
favourably with major CAS and algebraic integration packages.  \\

The computational burden of our method is low, as the core computational routine requires 
solving multiple systems of linear equations, and consequently our method should be tried before the 
more computationally expensive algorithms of Trager\cite{Trager1984}, Bronstein\cite{Bronstein1990}, 
Kauers\cite{Kauers2008}, or Miller\cite{Miller2012}.\\

Our method, relative to the algebraic case of the Risch-Bronstein-Trager algorithm, is very simple to 
implement. Our exemplar implementation in Mathematica is only a couple of hundred lines of code and 
relies heavily on \texttt{SolveAlways} for solving polynomial equations with undetermined coefficients. 
The implementation is available on github\cite{algebraic_github}. 

\bibliographystyle{abbrv}

\begin{thebibliography}{99}

\bibitem{Bronstein1990} Bronstein, M. (1990). ``Integration of Elementary Functions''. 
\textit{Journal of Symbolic Computation}. 9(2), pp. 117-173.

\bibitem{Bronstein1997} Bronstein, M. (1997). ``Symbolic Integration 1: Transcendental Functions''. Springer-Verlag.

\bibitem{Davenport1979} Davenport, J. (1979). ``Integration of algebraic functions'', \textit{EUROSAM `79: Proceedings of the International Symposium on Symbolic and Algebraic Computation. }pp. 415-425.

\bibitem{Euler} \url{https://en.wikipedia.org/wiki/Euler_substitution}

\bibitem{fricas_risch_status} \url{http://fricas-wiki.math.uni.wroc.pl/RischImplementationStatus}

\bibitem{Geddes1992} Geddes, K. Czapor, S. Labahn, G. (1992). ``Algorithms for Computer Algebra'', Springer US, 
ISBN \verb~978-0-7923-9259-0~.

\bibitem{Hardy1916} Hardy, G. (1916). \textit{The Integration of Functions of a Single Variable}. Cambridge 
University Press. Cambridge, England.

\bibitem{Kauers2008} Kauers, M. (2008). ``Integration of algebraic functions: a simple heuristic for finding the logarithmic part''. \textit{ISSAC `08}. pp. 133-140.

\bibitem{Miller2012} Miller, B. (2012). ``On the Integration of Elementary Functions: Computing the Logarithmic Part''. Thesis (Ph.D.) Texas Tech University, Dept. of Mathematics and Statistics.

\bibitem{Moses1967} Moses, J. (1967). ``Symbolic Integration''. MAC-TR-47, MIT, Cambridge, MA.

\bibitem{Prudnikov1986}	Prudnikov, A.P. Brychkov, A. Marichev, O.I. ``Integrals and Series, Volume 1, Elementary Functions''. CRC Press. edition 1. ISBN-10: 2881240976.

\bibitem{Rich2018} Rich, A. Scheibe, P. Abbasi, N. (2018). ``Rule-based integration: An extensive system of symbolic integration rules''. Journal of Open Source Software. 3.32, p1073 1-3.

\bibitem{Risch1969} Risch, R. (1969). ``The Problems of Integration in Finite Terms'', \textit{ Trans. AMS}. 139(1). pp. 167-189.

\bibitem{rubi_results} \url{https://rulebasedintegration.org/testResults}

\bibitem{Trager1984} Trager, B. (1984). ``Integration of algebraic functions''. Thesis (Ph.D.) Massachusetts Institute of Technology, Dept. of Electrical Engineering and Computer Science.

\bibitem{test_suite_github} \url{https://github.com/stblake/algebraic_integration/blob/master/comparisonIntegrands.m}

\bibitem{algebraic_github} \url{https://github.com/stblake/algebraic_integration/blob/master/AlgebraicIntegrateHeuristic.m}

\bibitem{fricas_sci.math.symbolic} \url{https://groups.google.com/forum/#!topic/sci.math.symbolic/zd0v05DNrNc}

\end{thebibliography}

\end{document}
